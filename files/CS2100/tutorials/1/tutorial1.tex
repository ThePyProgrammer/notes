\documentclass[a4paper, 12pt, addpoints]{exam}
\usepackage[utf8]{inputenc}
\usepackage[english]{babel}
\usepackage{amsmath}
\usepackage{tcolorbox}
\usepackage{graphicx}
\usepackage{wasysym}
\printanswers
\title{CS2100 \textbf{Tutorial 1}}
\author{Prannaya Gupta}
\date{$24^{\text{th}}$ August 2022}
\begin{document}

\maketitle


\section{Discussion Questions}
\begin{questions}
\question In 2’s complement representation, “sign extension” is used when we want to represent an $n$-bit signed integer as an $m$-bit signed integer, where $m > n$. We do this by copying the sign-bit of the $n$-bit signed $m – n$ times to the left of the $n$-bit number to create an $m$-bit number. So for example, we want to sign-extend \texttt{0b0110} to an 8-bit number. Here $n = 4$, $m = 8$, and thus we copy the sign but $m – n = 4$ times, giving us \texttt{0b00000110}. Similarly if we want to sign-extend \texttt{0b1010} to an 8-bit number, we would get \texttt{0b11111010}. Show that IN GENERAL sign extension is value-preserving. For example, \texttt{0b00000110 = 0b0110} and \texttt{0b11111010 = 0b1010}.

\end{questions}
\newpage

\section{Practice Questions}

\begin{questions}
\question 

\end{questions}


\end{document}
