\documentclass[11pt]{article}

% basic packages
\usepackage[margin=1in]{geometry}
\usepackage[pdftex]{graphicx}
\usepackage{amsmath,amssymb,amsthm}
\usepackage{notes}
\usepackage{lipsum}

% page formatting
\usepackage{fancyhdr}
\pagestyle{fancy}
\usepackage{hyperref}
\usepackage{tcolorbox}

\renewcommand{\sectionmark}[1]{\markright{\textsf{\arabic{section}. #1}}}
\renewcommand{\subsectionmark}[1]{}
\lhead{\textbf{\thepage} \ \ \nouppercase{\rightmark}}
\chead{}
\rhead{}
\lfoot{}
\cfoot{}
\rfoot{}
\setlength{\headheight}{14pt}

\linespread{1.03} % give a little extra room
\setlength{\parindent}{0.2in} % reduce paragraph indent a bit
\setcounter{secnumdepth}{2} % no numbered subsubsections
\setcounter{tocdepth}{2} % no subsubsections in ToC

\begin{document}

% make title page
\thispagestyle{empty}
\bigskip \
\vspace{0.1cm}

\begin{center}
{\fontsize{22}{22} \selectfont Compiled Notes for}
\vskip 16pt
{\fontsize{36}{36} \selectfont \bf \sffamily PC5132}
\vskip 24pt
{\fontsize{18}{18} \selectfont \rmfamily Prannaya Gupta} 
\vskip 6pt
{\fontsize{14}{14} \selectfont \ttfamily prannayagupta@gmail.com} 
\vskip 24pt
\end{center}

{\parindent0pt \baselineskip=15.5pt Heve fun reading this man.}

% make table of contents
\newpage
\microtoc
\newpage

% main content
\section{Nuclear Binding Energy}
\begin{align*}
    \text{Binding Energy, B.E.} &= -\Delta mc^2 \\
    \text{Binding Energy per Nucleon, B.E.}/A &= \frac{-\Delta mc^2}{A}
\end{align*}





\section{Properties of Particles}

\subsection{Properties of a Photon}
\begin{align*}
    E &= hf = \frac{hc}\lambda \\
    p &= \frac{h}\lambda
\end{align*}

\subsection{Properties of a Free Particle}

\subsubsection{Wave-Particle Duality - De Broglie Wavelength}

$$\lambda = \frac{h}p$$

\subsubsection{Energy}
\begin{align*}
E^2 &= (pc)^2 + (mc^2)^2
\end{align*}


\section{Photoelectric Effect}

\begin{align*}
    E = hf &= K + \phi \\
    &= eV_s + hf_o
\end{align*}

\section{Emission Spectra}
\subsection{Rydberg's Formula}

$$\frac{1}\lambda = R_\infty\left(\frac{1}{m^2} - \frac{1}{n^2} \right) \quad \implies \quad \lambda = \frac{m^2n^2}{(n^2-m^2)R_\infty}$$

Here $R_\infty$ is the Rydberg Constant.

\subsection{Excitation Energy}

$$E = -hcR_\infty \frac{1}{n^2} = -\frac{\text{13.6 eV}}{n^2}$$

\section{Heisenberg}

Walter White.

$$\Delta x \Delta p \geq \frac{h}{4\pi} = \frac{\hbar}{2}$$

\subsection{Examples}

\subsubsection{Wave Particle Duality Practice Question 1}

A student on a ladder drops small pellets towards a point target on the floor. Show that, according to the uncertainty principle, the average miss distance must be at least

$$\Delta x_f = \left(\frac{2\hbar}{m} \right)^\frac{1}{2} \left(\frac{2H}{g} \right)^\frac{1}{4}$$

where $H$ is the initial height of each pellet above the floor and m is the mass of each pellet. Assume that the spread in impact points is given by $\Delta x_f = \Delta x_i + (\Delta v_x) t$.

\begin{tcolorbox}
\textbf{Solution: (at least I think)}

Firstly, note that this is essentially a projectile motion question. Notably, you know $S_y$, $a_y$ and $v_y$, so can get an expression for $t$:
\begin{align*}
    S_y &= v_yt + \frac{1}2 a_y t^2 \\
    -H &= \frac{1}2 (-g) t^2 \\
    t &= \sqrt{\frac{2H}{g}}
\end{align*}

From here, you can get an expression for $\Delta v_f$ in terms of $\Delta v_i$ and $\Delta v_x$:
\begin{align*}
    \Delta x_f &= \Delta x_i + (\Delta v_x)t \\
    &= \Delta x_i + (\Delta v_x) \sqrt{\frac{2H}{g}}
\end{align*}

By Walter White's Uncertainty Principle, we have the following fact too:
\begin{align*}
    \Delta x_i \Delta p &= \frac{\hbar}{2} \\
    \Delta x_i &= \frac{\hbar}{2m\Delta v_x}
\end{align*}

Thus, you substitute $\Delta x_i$ in terms of $\Delta v_x$ as follows:
\begin{align*}
    \Delta x_f &= \frac{\hbar}{2m\Delta v_x} + (\Delta v_x) \sqrt{\frac{2H}{g}}
\end{align*}

To minimize, we differentiate the above with respect to $\Delta v_x$, and let this derivative be 0:
\begin{align*}
    \frac{d(\Delta x_f)}{d(\Delta v_x)} &= -\frac{\hbar}{2m(\Delta v_x)^2} + \sqrt{\frac{2H}{g}} \\
    0 &= \sqrt{\frac{2H}{g}} -\frac{\hbar}{2m(\Delta v_x)^2} \\
    \Delta v_x &= \sqrt{\frac{\hbar}{2m}} \left(\frac{2H}{g} \right)^{-\frac{1}4} \\
    \Delta x_f &= \sqrt{\frac{\hbar}{2m}} \left(\frac{2H}{g} \right)^{\frac{1}4} + \sqrt{\frac{\hbar}{2m}} \left(\frac{2H}{g} \right)^{\frac{1}4} \\ &= \sqrt{\frac{2\hbar}{m}} \left(\frac{2H}{g} \right)^{\frac{1}4}
\end{align*}

\end{tcolorbox}

\end{document}