\documentclass[a4paper, 12pt, addpoints]{exam}
\usepackage[utf8]{inputenc}
\usepackage[english]{babel}
\usepackage{amsmath}
\usepackage{tcolorbox}
\usepackage{graphicx}
\usepackage{wasysym}
\printanswers
\title{Chapter 6 \textbf{Photoelectric Effect}}
\author{Prannaya Gupta}
\date{$23^{\text{rd}}$ August 2022}
\begin{document}

\maketitle


\begin{tcolorbox}
Unless otherwise stated:
\begin{align*}
\text{Stefan-Boltzmann Constant, }\sigma &= 5.670 \times 10^{-8}\text{ W m${}^{-2}$ K${}^{-4}$} \\
\text{Wien's Displacement Constant, }b &= 2.898 \times 10^{-3} \text{ m K} \\
\text{Boltzmann Constant, }k_\text{B} &= 1.381 \times 10^{-23}\text{ m${}^2$ kg s${}^{-2}$ K${}^{-1}$} \\
\text{Planck's Constant, }h &= 6.626 \times 10^{-34}\text{ J s} \\
\text{Speed of Light, }c &= 3.00 \times 10^8 \text{ m/s}
\end{align*}
\end{tcolorbox}
\newpage
\section{Discussion Questions}
\begin{questions}
\question The average threshold of dark-adapted (scotopic) vision is $4.00 \times 10^{-11}\text{ W m}^{-2}$ at a central wavelength of 500 nm. If light with this intensity and wavelength enters the eye and the pupil is open to a diameter of 7.0 mm, how many photons enter the eye per second?

\begin{tcolorbox}
\textbf{Solution: }
\begin{align*}
    I &= \frac{Nhf}{tA} \\
    \frac{N}{t} &= \frac{I(\pi \times d^2)\lambda}{4hc} \\
    &= \frac{4.00 \times 10^{-11} \times \pi \times (7.0 \times 10^{-3})^2 \times 500 \times 10^{-9}}{4 \times 6.626 \times 10^{-34} \times 3.0 \times 10^8} \\
    &= 3872\text{ photons} \\
    &\approx \textbf{3800 photons}\text{ (2 s.f.)}
\end{align*}
\end{tcolorbox}

\question In an experiment, EM radiation of wavelength 480 nm and with an intensity of 100 W m${}^{-2}$ was incident on a potassium surface. When an area of 12 mm${}^2$ was illuminated, a photocurrent of $4.5\times 10^{-10}$ A was measured.

\begin{parts}
\part Calculate the rate of incidence of photons on the metal surface.
\begin{tcolorbox}
\textbf{Solution: }
\begin{align*}
I &= \frac{Nhf}{tA} \\
\frac{N}{t} &= \frac{IA\lambda}{hc} \\
&= \frac{100 \times 12 \times 10^{-6} \times 480 \times 10^{-9}}{6.626\times 10^{-34} \times 3.0 \times 10^8} \\
&= 2.89768 \times 10^{15} \text{ photons per second} \\
&\approx \bf{2.9 \times 10^{15}}\textbf{ photons per second}\text{ (2 s.f.)}
\end{align*}
\end{tcolorbox}

\part Calculate the rate of emission of electrons.
\begin{tcolorbox}
\textbf{Solution: }
\begin{align*}
    I &= \frac{Q}{t} \\
    &= \frac{Nq_e}{t} \\
    \frac{N}{t} &= \frac{I}{q_e} \\
    &= \frac{4.5\times 10^{-10}}{1.60 \times 10^{-19}} \\
    &= 2.8125 \times 10^{9} \text{ electrons per second} \\
    &\approx \bf{2.8 \times 10^{9}}\textbf{ electrons per second}\text{ (2 s.f.)}
\end{align*}
\end{tcolorbox}
\end{parts}

\question A student claims that she is going to eject electrons from a piece of metal by placing a radio transmitter antenna adjacent to the metal and sending a string FM radio signal into the antenna. The work function of a metal is typically a few electron volts. Will this work?

\begin{tcolorbox}
\textbf{Solution: }Very likely not.

The energy of a radio wave is really very small, and it likely can't surpass a few electron volts.
\end{tcolorbox}

\question A student studying the photoelectric effect from two different metals records the following information when they are illuminated by the same source:

\begin{enumerate}
\item the stopping potential for photoelectrons released from metal 1 is 1.41 V larger than that for metal 2 and
\item the threshold frequency for metal 1 is 60.0\% that f metal 2.
\end{enumerate}

Determine the work function for each metal.

\begin{tcolorbox}
\textbf{Solution: }\\
We have the following simultaneous expressions:
\begin{align*}
    E_1 &= hf_{o, 1} + q_eV_{s, 1} \\
    E_2 &= hf_{o, 2} + q_eV_{s, 2} \\
    0.6 hf_{o, 2} + q_e(V_{s, 2} + 1.41) &= hf_{o, 2} + q_e (V_{s,2}) \\
    q_e (1.41) &= 0.4 \phi_2 \\
    \phi_2 &= \frac{1.6\times 10^{-19} \times 1.41}{0.4} \\
    \phi_2 &= 3.525\text{ eV} \\
    &\approx \textbf{3.5 eV} \\
    \phi_1 &= 0.6 \phi_2 \\
    &= 2.115\text{ eV} \\
    &\approx \textbf{2.1 eV}
\end{align*}

\end{tcolorbox}


\question The \textbf{threshold wavelength} of mercury is 276 nm.
\begin{parts}
\part Define \textit{threshold wavelength}.
\begin{tcolorbox}
The threshold wavelength is the maximum wavelength of an EM radiation required for 
\end{tcolorbox}
\end{parts}


\question A very weak beam
\end{questions}
\newpage

\section{Practice Questions}

\begin{questions}
\question 

\end{questions}


\end{document}
