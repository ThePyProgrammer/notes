\documentclass[a4paper, 12pt, addpoints]{exam}
\usepackage[utf8]{inputenc}
\usepackage[english]{babel}
\usepackage{amsmath}
\usepackage{tcolorbox}
\usepackage{graphicx}
\usepackage{wasysym}
\printanswers
\title{Chapter 9 \textbf{Wave-Particle Duality}}
\author{Prannaya Gupta}
\date{$24^{\text{th}}$ October 2022}
\begin{document}

\maketitle


\begin{tcolorbox}
Unless otherwise stated:
\begin{align*}
\text{Mass of an Electron, }m_e &= 9.11 \times 10^{-31}\text{ kg} \\
\text{Mass of a Proton, }m_p &= 1.67262 \times 10^{-27}\text{ kg} \\
\text{Planck's Constant, }h &= 6.626 \times 10^{-34}\text{ J s} \\
\text{Speed of Light, }c &= 3.00 \times 10^8 \text{ m/s}
\end{align*}
\end{tcolorbox}
\newpage


\section{Discussion Questions}
\begin{questions}
\question Determine:
\begin{parts}
    \part the momentum of photons with a frequency of $4.0 \times 10^{14}$ Hz.
    \begin{tcolorbox}
        \textbf{Solution:}
        $$p = \frac{h}\lambda = \frac{hf}{c} = 8.8347 \times 10^{-28}\text{ N s}$$
    \end{tcolorbox}
    
    \part the momentum of photons with a wavelength of 400 nm.
    \begin{tcolorbox}
        \textbf{Solution:}
        $$p = \frac{h}\lambda = \frac{h}{400 \times 10^{-9}} = 1.6565 \times 10^{-27}\text{ N s}$$
    \end{tcolorbox}
    
    \part A beam of light of wavelength $\lambda$ is totally reflected at normal incidence by a plane mirror. The intensity of light is such that the photons hit the mirror at a rate $n$ per second. Given that the Planck constant is $h$, show that the force exerted on the mirror by this beam is $2nh/\lambda$.
    \begin{tcolorbox}
        \textbf{Solution:}
        The change in momentum per photon is $2p = 2h/\lambda$. Thus, the total change in momentum over some time interval $t$ is $2nht/\lambda$. Thus, $F = p/t = 2nh/\lambda$.
    \end{tcolorbox}
\end{parts}

\question
\begin{parts}
    \part Determine the de Broglie wavelength of 
    \begin{subparts}
        \subpart a bullet of mass 30 g, and moving with speed of 300 $\text{m s}^{-1}$.
        \begin{tcolorbox}
            \textbf{Solution:}
            $$p = mv = (0.030)(300) = 9 \text{ Ns} \implies \lambda = \frac{h}p = 7.3622 \times 10^{-35}\text{ m}$$
        \end{tcolorbox}

        \subpart an electron that has been accelerated from rest through a potential difference of 0.75 kV.
        \begin{tcolorbox}
            \textbf{Solution:}
            \begin{align*}
                E_f &= E_i + \Delta E \\
                (m_e c^2)^2 + (pc)^2 &= (m_e c^2 + \Delta E)^2 \\
                p &= \sqrt{ \left(2m_ec + \frac{\Delta E}{c}\right) \times \frac{\Delta E}{c}  } \\
                \lambda &= \frac{h}{\sqrt{ \left(2m_ec + \frac{q_e V}{c}\right) \times \frac{q_e V}{c}  }} \\
                &= 4.47948 \times 10^{-11}\text{ m}
            \end{align*}
        \end{tcolorbox}
    \end{subparts}
    \part In an electron diffraction experiment, the wavelength associated with an electron beam is determined to be 0.14 nm.
    \begin{subparts}
        \subpart Find the momentum of an electron in the beam.
        \begin{tcolorbox}
            \textbf{Solution:}
            $$p = \frac{h}\lambda = \frac{6.626 \times 10^{-34}}{0.14 \times 10^{-9}} = 4.7329 \times 10^{-24} \text{ N s}$$
        \end{tcolorbox}
        \subpart If the electrons are initially at rest, through what potential difference should they be accelerated to acquire this momentum.
        \begin{tcolorbox}
            \textbf{Solution:}
            \begin{align*}
                \Delta E &= \sqrt{(m_ec^2)^2 + (pc)^2} - m_ec^2 \\
                V &= \frac{1}{q_e} \left( \sqrt{(m_ec^2)^2 + (pc)^2} - m_ec^2 \right) \\\
                &= 76.834 \text{ V}
            \end{align*}
        \end{tcolorbox}
    \end{subparts}
\end{parts}

\question
\begin{parts}
    \part An electron has kinetic energy 3.00 eV. Find its wavelength.
    \begin{tcolorbox}
        \textbf{Solution:}
        $$p = \sqrt{2m_eE_k} \implies \lambda = \frac{h}{\sqrt{2m_eE_k}} = 7.0853 \times 10^{10}\text{ m}$$
    \end{tcolorbox}
    \part A photon has energy 3.00 eV. Find it's wavelength.
    \begin{tcolorbox}
        \textbf{Solution:}
        $$\lambda = \frac{hc}E = 4.14125 \times 10^{-7}\text{ m} \approx 414\text{ nm}$$
    \end{tcolorbox}
\end{parts}

\question The uncertainty in the position and velocity of a particle are $10^{-11}$ m and $7.9 \times 10^2$ m/s respectively. By considering Heisenberg's Uncertainty Principle, suggest what this particle may be.
\begin{tcolorbox}
    \textbf{Solution:}
    \begin{align*}
        \Delta x \Delta p &\geq \frac{\hbar}{2} \\
        m &\geq \frac{\hbar}{2\Delta x \Delta v} \\
        &= 6.67443 \times 10^{-27}\text{ kg} \\
        m &\geq 4.0207\text{ u}.
    \end{align*}
    Thus, it is likely to be an Alpha Particle.
\end{tcolorbox}

\end{questions}
\newpage

\section{Practice Questions}

\begin{questions}
\question A student on a ladder drops small pellets towards a point target on the floor. 
\begin{parts}
    \part Show that, according to the uncertainty principle, the average miss distance must be at least
    $$\Delta x_f = \left(\frac{2\hbar}{m} \right)^\frac{1}{2} \left(\frac{2H}{g} \right)^\frac{1}{4}$$
    
    where $H$ is the initial height of each pellet above the floor and m is the mass of each pellet. Assume that the spread in impact points is given by $\Delta x_f = \Delta x_i + (\Delta v_x) t$.
    
    \begin{tcolorbox}
    \textbf{Solution:}
    
    Firstly, note that this is essentially a projectile motion question. Notably, you know $S_y$, $a_y$ and $v_y$, so can get an expression for $t$:
    $$S_y = v_yt + \frac{1}2 a_y t^2 \implies -H = \frac{1}2 (-g) t^2 \implies t = \sqrt{\frac{2H}{g}}$$
    
    From here, you can get an expression for $\Delta v_f$ in terms of $\Delta v_i$ and $\Delta v_x$:
    $$\Delta x_f = \Delta x_i + (\Delta v_x)t = \Delta x_i + (\Delta v_x) \sqrt{\frac{2H}{g}}$$
    
    By Heisenberg's Uncertainty Principle, we have the following fact too:
    $$\Delta x_i \Delta p = \frac{\hbar}{2} \implies \Delta x_i = \frac{\hbar}{2m\Delta v_x}$$
    
    Thus, you substitute $\Delta x_i$ in terms of $\Delta v_x$ as follows:
    \begin{align*}
        \Delta x_f &= \frac{\hbar}{2m\Delta v_x} + (\Delta v_x) \sqrt{\frac{2H}{g}}
    \end{align*}
    
    To minimize, we differentiate the above with respect to $\Delta v_x$, and let this derivative be 0:
    \begin{align*}
        \frac{d(\Delta x_f)}{d(\Delta v_x)} &= -\frac{\hbar}{2m(\Delta v_x)^2} + \sqrt{\frac{2H}{g}} = 0 \\
        \Delta v_x &= \sqrt{\frac{\hbar}{2m}} \left(\frac{2H}{g} \right)^{-\frac{1}4} \\
        \Delta x_f &= \sqrt{\frac{\hbar}{2m}} \left(\frac{2H}{g} \right)^{\frac{1}4} + \sqrt{\frac{\hbar}{2m}} \left(\frac{2H}{g} \right)^{\frac{1}4} \\ &= \sqrt{\frac{2\hbar}{m}} \left(\frac{2H}{g} \right)^{\frac{1}4}
    \end{align*}
    \end{tcolorbox}
    \newpage
    \part If $H = 2.00$ m and $m = 0.500$ g, what is $\Delta x_f$?
    \begin{tcolorbox}
        \textbf{Solution:}
        $$\Delta x_f = \sqrt{\frac{6.626 \times 10^{-34}}{5.00\pi \times 10^{-4}}} \left(\frac{4.00}{9.81} \right)^{\frac{1}4} = 5.18995 \times 10^{-16}\text{ m}$$
    \end{tcolorbox}
\end{parts}

\question
\begin{parts}
    \part Show that the kinetic energy of a nonrelativistic particle can we written in terms of its momentum as $K = p^2/2m$.
    \begin{tcolorbox}
        \textbf{Solution:}
        $$E_k = \frac{1}2 mv^2 = \frac{(mv)^2}{2m} = \frac{p^2}{2m}$$
    \end{tcolorbox}

    \part Use the result from (a) to find the minimum kinetic energy of a proton confined within a nucleus having a diameter of $1.00 \times 10^{-15}$ m.
    \begin{tcolorbox}
        \textbf{Solution:}
        By Heisenberg's Uncertainty Principle,
        $$\Delta x \Delta p \geq \frac{\hbar}{2} \implies \Delta p \geq \frac{\hbar}{2\Delta x}$$
        Notably, $\Delta x = D = 1.00 \times 10^{-15}$ m. Using the result from in (a), we get the following:
        \begin{align*}
            \Delta p &\geq \frac{\hbar}{2\Delta x} \\
            E_\text{min} &= \frac{1}{2m}\left(\frac{\hbar}{2\Delta x} \right)^2 \\
            &= 5.20256 \text{ MeV}
        \end{align*}
    \end{tcolorbox}
\end{parts}

\question An electron ($m_e = 9.11 \times 10^{-31}\text{ kg}$) and a bullet ($m = 0.0200\text{ kg}$) each have a velocity of magnitude of 500 m/s, accurate to within 0.0100\%. Within what limits could we determine the position of the objects along the direction of the velocity.

\begin{tcolorbox}
    \textbf{Solution:}\\
    For Electron:
    $$\Delta x = \frac{h}{4\pi \Delta p} = \frac{h}{4\pi m_e (0.05)} = 1.15759 \text{ mm}$$
    For Bullet:
    $$\Delta x = \frac{h}{4\pi \Delta p} = \frac{h}{4\pi m (0.05)} = 5.27280 \times 10^{-32}\text{ m}$$
\end{tcolorbox}
\end{questions}


\end{document}
