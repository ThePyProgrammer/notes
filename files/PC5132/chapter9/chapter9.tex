\documentclass[a4paper, 12pt, addpoints]{exam}
\usepackage[utf8]{inputenc}
\usepackage[english]{babel}
\usepackage{amsmath}
\usepackage{tcolorbox}
\usepackage{graphicx}
\usepackage{wasysym}
\printanswers
\title{Chapter 9 \textbf{Wave-Particle Duality}}
\author{Prannaya Gupta}
\date{$24^{\text{th}}$ October 2022}
\begin{document}

\maketitle


% \begin{tcolorbox}
% Unless otherwise stated:
% \begin{align*}
% \text{Stefan-Boltzmann Constant, }\sigma &= 5.670 \times 10^{-8}\text{ W m${}^{-2}$ K${}^{-4}$} \\
% \text{Wien's Displacement Constant, }b &= 2.898 \times 10^{-3} \text{ m K} \\
% \text{Boltzmann Constant, }k_\text{B} &= 1.381 \times 10^{-23}\text{ m${}^2$ kg s${}^{-2}$ K${}^{-1}$} \\
% \text{Planck's Constant, }h &= 6.626 \times 10^{-34}\text{ J s} \\
% \text{Speed of Light, }c &= 3.00 \times 10^8 \text{ m/s}
% \end{align*}
% \end{tcolorbox}
% \newpage


\section{Discussion Questions}
\begin{questions}
\question Determine:
\begin{parts}
    \part the momentum of photons with a frequency of $4.0 \times 10^{14}$ Hz.
    \begin{tcolorbox}
        \textbf{Solution:}
        $$p = \frac{h}\lambda = \frac{hf}{c} = 8.8347 \times 10^{-28}\text{ N s}$$
    \end{tcolorbox}
    
    \part the momentum of photons with a wavelength of 400 nm.
    \begin{tcolorbox}
        \textbf{Solution:}
        $$p = \frac{h}\lambda = \frac{h}{400 \times 10^{-9}} = 1.6565 \times 10^{-27}\text{ N s}$$
    \end{tcolorbox}
    
    \part A beam of light of wavelength $\lambda$ is totally reflected at normal incidence by a plane mirror. The intensity of light is such that the photons hit the mirror at a rate $n$ per second. Given that the Planck constant is $h$, show that the force exerted on the mirror by this beam is $2nh/\lambda$.
    \begin{tcolorbox}
        \textbf{Solution:}
        The change in momentum per photon is $2p = 2h/\lambda$. Thus, the total change in momentum over some time interval $t$ is $2nht/\lambda$. Thus, $F = p/t = 2nh/\lambda$.
    \end{tcolorbox}
\end{parts}

\question
\begin{parts}
    \part Determine
    \begin{subparts}
        \subpart the de Broglie wavelength of a bullet of mass 30 g, and moving with speed of 300 $\text{m s}^{-1}$.
        \begin{tcolorbox}
            \textbf{Solution:}
            $$p = mv = (0.030)(300) = 9 \text{ Ns} \implies \lambda = \frac{h}p = 7.3622 \times 10^{-35}\text{ m}$$
        \end{tcolorbox}

        \subpart the de Broglie wavelength of an electron that has been accelerated from rest through a potential difference of 0.75 kV.
        \begin{tcolorbox}
            \textbf{Solution:}
            $$p = \sqrt{2m_eE_K} \implies \lambda = \frac{h}{\sqrt{2m_eE_K}} = $$
        \end{tcolorbox}
    \end{subparts}
\end{parts}
\end{questions}
\newpage

\section{Practice Questions}

\begin{questions}
\question A student on a ladder drops small pellets towards a point target on the floor. Show that, according to the uncertainty principle, the average miss distance must be at least

$$\Delta x_f = \left(\frac{2\hbar}{m} \right)^\frac{1}{2} \left(\frac{2H}{g} \right)^\frac{1}{4}$$

where $H$ is the initial height of each pellet above the floor and m is the mass of each pellet. Assume that the spread in impact points is given by $\Delta x_f = \Delta x_i + (\Delta v_x) t$.

\begin{tcolorbox}
\textbf{Solution:}

Firstly, note that this is essentially a projectile motion question. Notably, you know $S_y$, $a_y$ and $v_y$, so can get an expression for $t$:
$$S_y = v_yt + \frac{1}2 a_y t^2 \implies -H = \frac{1}2 (-g) t^2 \implies t = \sqrt{\frac{2H}{g}}$$

From here, you can get an expression for $\Delta v_f$ in terms of $\Delta v_i$ and $\Delta v_x$:
$$\Delta x_f = \Delta x_i + (\Delta v_x)t = \Delta x_i + (\Delta v_x) \sqrt{\frac{2H}{g}}$$

By Walter White's Uncertainty Principle, we have the following fact too:
$$\Delta x_i \Delta p = \frac{\hbar}{2} \implies \Delta x_i = \frac{\hbar}{2m\Delta v_x}$$

Thus, you substitute $\Delta x_i$ in terms of $\Delta v_x$ as follows:
\begin{align*}
    \Delta x_f &= \frac{\hbar}{2m\Delta v_x} + (\Delta v_x) \sqrt{\frac{2H}{g}}
\end{align*}

To minimize, we differentiate the above with respect to $\Delta v_x$, and let this derivative be 0:
\begin{align*}
    \frac{d(\Delta x_f)}{d(\Delta v_x)} &= -\frac{\hbar}{2m(\Delta v_x)^2} + \sqrt{\frac{2H}{g}} = 0 \\
    \Delta v_x &= \sqrt{\frac{\hbar}{2m}} \left(\frac{2H}{g} \right)^{-\frac{1}4} \\
    \Delta x_f &= \sqrt{\frac{\hbar}{2m}} \left(\frac{2H}{g} \right)^{\frac{1}4} + \sqrt{\frac{\hbar}{2m}} \left(\frac{2H}{g} \right)^{\frac{1}4} \\ &= \sqrt{\frac{2\hbar}{m}} \left(\frac{2H}{g} \right)^{\frac{1}4}
\end{align*}

\end{tcolorbox}

\end{questions}


\end{document}
