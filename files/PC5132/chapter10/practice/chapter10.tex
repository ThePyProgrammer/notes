\documentclass[a4paper, 12pt, addpoints]{exam}
\usepackage[utf8]{inputenc}
\usepackage[english]{babel}
\usepackage{amsmath}
\usepackage{tcolorbox}
\usepackage{graphicx}
\usepackage{wasysym}
\printanswers
\title{Chapter 10 \textbf{Compton Scattering}}
\author{Prannaya Gupta}
\date{$25^{\text{th}}$ October 2022}
\begin{document}

\maketitle


\begin{tcolorbox}
Unless otherwise stated:
\begin{align*}
\text{Mass of an Electron, }m_e &= 9.11 \times 10^{-31}\text{ kg} \\
\text{Mass of a Proton, }m_p &= 1.67262 \times 10^{-27}\text{ kg} \\
\text{Planck's Constant, }h &= 6.626 \times 10^{-34}\text{ J s} \\
\text{Speed of Light, }c &= 3.00 \times 10^8 \text{ m/s}
\end{align*}
\end{tcolorbox}
\newpage


\section{Discussion Questions}
\begin{questions}
\question X-rays having an energy of 350 keV undergo Compton scattering from a target. The scattered rays are detected at 45.0\textdegree relative to the incident rays. Find.
\begin{parts}
    \part the Compton shift at this angle
    \begin{tcolorbox}
        \textbf{Solution:}
        $$\Delta \lambda = \frac{h}{m_ec}(1-\cos 45^\circ) = 7.10103 \times 10^{-13}\text{ m}$$
    \end{tcolorbox}
    \part the energy of the scattering x-ray, and
    \begin{tcolorbox}
        \textbf{Solution:}
        First, we need to get the initial $\lambda_i$:
        $$E_i = \frac{hc}{\lambda_i} \implies \lambda_i = \frac{hc}{E_i}$$
        Thus, we know the following:
        $$E_f = \frac{hc}{\lambda_f} = \frac{hc}{\frac{hc}{E_i} + \Delta \lambda} = \frac{1}{\frac{1}{E_i} + \frac{\Delta \lambda}{hc}} = 291.654\text{ keV} \approx 292\text{ keV} $$
    \end{tcolorbox}
    \part the energy of the recoiling electron.
    \begin{tcolorbox}
        $$E_e = \Delta E = 350 - 292 = 58\text{ keV}$$
    \end{tcolorbox}
\end{parts}

\question A 0.70 MeV photon is scattered by a free electron initially at rest such that the scattering angle of the scattered photon, $\theta$ is equal to that of the scattered electron $\varphi$.

\begin{parts}
    \part Determine the angles $\theta$ and $\varphi$.
    \begin{tcolorbox}
        \textbf{Solution:} \\
        Note that since both angles are the same, the magnitudes of the momentum should be equal. The initial momentum is given by:
        $$p = \frac{E}{c} = 3.73333 \times 10^{-22}\text{ N s} = 2p_f \cos\theta$$
        Note that by Compton Scattering, 
        $$\Delta \lambda = \frac{h}{m_ec}(1-\cos \theta) \implies \frac{1}{p_f} = \frac{1}{p} + \frac{1-\cos \theta}{m_e c}$$
        Hence,
        \begin{align*}
            \frac{2\cos\theta - 1}{p} &= \frac{1-\cos \theta}{m_e c} \\
            p(1 - \cos\theta) &= m_e c(2\cos\theta - 1) \\
            (2m_e c + p)\cos\theta &= p + m_e c \\
            \theta = \varphi &= 45.339^\circ
        \end{align*}
    \end{tcolorbox}

    \part Determine the energy and momentum of the scattered photon.
    \begin{tcolorbox}
        \textbf{Solution:}
        $$p_f = \frac{p}{2\cos\theta} = 2.65561 \times 10^{-22}\text{ N s}$$
        $$E_f = p_f c = 0.497927\text{ MeV} $$
    \end{tcolorbox}
    \part Determine the kinetic energy and momentum of the scattered electron.
    \begin{tcolorbox}
        \textbf{Solution:}
        $$p_f = \frac{p}{2\cos\theta} = 2.65561 \times 10^{-22}\text{ N s}$$
        $$E = E_i - E_f = 0.70 - 0.497927 = 0.202073\text{ MeV}$$
    \end{tcolorbox}
\end{parts}

\question In a Compton Scattering experiment, x-ray photons scattered at 60\textdegree change by 2\% in wavelength.
\begin{parts}
    \part Calculate the wavelength of the incident x-rays.
    \begin{tcolorbox}
        \textbf{Solution:}
        $$(2\%)\lambda = \frac{h}{m_ec}(1 - \cos 60^\circ) \implies \lambda = \frac{25h}{m_ec} = 6.0611\times 10^{-11}\text{ m}$$
    \end{tcolorbox}
    \part What is the \textbf{maximum} possible wavelength of the output x-ray beam?
    \begin{tcolorbox}
        $$\lambda_\text{max} = \frac{hc}{E}$$
    \end{tcolorbox}
\end{parts}

\question An electron (rest energy 0.511 MeV) moves with speed 0.800 c. Determine:
\begin{parts}
    \part its total energy,
    \begin{tcolorbox}
        $$E = \sqrt{(pc)^2 + E_\text{rest}^2} = E_\text{rest} \sqrt{1 + \beta^2} = $$
    \end{tcolorbox}
\end{parts}

\end{questions}
\newpage

\section{Practice Questions}

\begin{questions}
\question 
\end{questions}


\end{document}
