\documentclass[a4paper, 12pt, addpoints]{exam}
\usepackage[utf8]{inputenc}
\usepackage[english]{babel}
\usepackage{amsmath}
\usepackage{tcolorbox}
\usepackage{graphicx}
\usepackage{wasysym}
\printanswers
\title{Singapore Physics Olympiad \textbf{2022}}
\author{Prannaya Gupta}
\date{$22^{\text{nd}}$ August 2022}
\begin{document}

\maketitle


\begin{tcolorbox}
\textbf{Solution:}\\
We know that for a standing wave, $L = n\frac{\lambda}{2}$. We can derive an expression by the velocity function of a string wave.
\begin{align*}
v &= \sqrt{\frac{T}{\mu}} \\
\frac{\lambda}2 &= \frac{1}{2f}\sqrt{\frac{T}{\mu}} = \frac{1}{2f} \sqrt{\frac{mg}{\mu}} \\
n &= 2fL \sqrt{\frac{\mu}{mg}} \\
\mu &= mg \left(\frac{n}{2fL} \right)^2
\end{align*}

From this expression, we know that mass is inversely proportional to $n^2$. Hence, we have the following:
\begin{align*}
\frac{n_1}{n_2} &= \sqrt{\frac{447.0}{286.1}} = 1.24996 \approx \frac{5}{4}
\end{align*}

Thus we have the values for $n_1$ and $n_2$ (which are both supposed to be integers). Thus,
\begin{align*}
\mu &= m_1 g \left(\frac{n_1}{2fL} \right)^2 \\
&= 286.1 \times 9.81 \times \left(\frac{5}{2\times 120 \times 1.20} \right)^2 \\
&= \textbf{0.84594} 
\end{align*}
\end{tcolorbox}


\end{document}
