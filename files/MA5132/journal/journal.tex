\documentclass[12pt]{exam}
\usepackage{tikz}
\usetikzlibrary{trees}
\usepackage[a4paper, margin=0.3in]{geometry}
\usepackage{tcolorbox}
\usepackage{amsmath}
\usepackage{float}

\makeatletter
\newcommand{\@giventhatstar}[2]{\left(#1\;\middle|\;#2\right)}
\newcommand{\@giventhatnostar}[3][]{#1(#2\;#1|\;#3#1)}
\newcommand{\giventhat}{\@ifstar\@giventhatstar\@giventhatnostar}
\makeatother

\setlength{\columnsep}{1cm}
\sectionfont{\fontsize{14}{15}\selectfont}
\subsectionfont{\fontsize{12}{15}\selectfont}

\title{\Large MA5132 Journal}
\author{\Small Prannaya Gupta (M22504)}
\date{}
\begin{document}
\maketitle
% Set the overall layout of the tree
\tikzstyle{level 1}=[level distance=3.5cm, sibling distance=3.5cm]
\tikzstyle{level 2}=[level distance=3.5cm, sibling distance=2cm]

% Define styles for bags and leafs
\tikzstyle{bag} = [text width=4em, text centered]
\tikzstyle{end} = [circle, minimum width=3pt,fill, inner sep=0pt]
\vspace{-1.5cm}
\section{Article 1}
\begin{questions}
\question Using the information given in Question 1 (see journal), draw a tree diagram to represent the probabilities given.

\begin{tcolorbox}
    Let $B$ denote the event of "Having Breast Cancer" and $T$ denote the event of "Testing Positive". Let the superscript $A^c$ denote the complement of an event $A$.
    
    % The sloped option gives rotated edge labels. Personally
    % I find sloped labels a bit difficult to read. Remove the sloped options
    % to get horizontal labels. 
    \begin{tikzpicture}[grow=right]
    \node[bag] {All Women}
        child {
            node[bag] {Have Breast Cancer, $B$}        
                child {
                    node[end, label=right:
                        {$P(B\cap T^c)=0.01 \cdot 0.1 = 0.001$}] {}
                    edge from parent
                    node[above] {$T^c$}
                    node[below]  {$0.1$}
                }
                child {
                    node[end, label=right:
                        {$P(B\cap T)= 0.01 \cdot 0.9 = 0.009$}] {}
                    edge from parent
                    node[above] {$T$}
                    node[below]  {$0.9$}
                }
                edge from parent 
                node[above] {$B$}
                node[below]  {$0.01$}
        }
        child {
            node[bag] {Don't have Breast Cancer, $B^c$}        
            child {
                    node[end, label=right:
                        {$P(B^c\cap T^c)=0.91 \cdot 0.99 = 0.9009$}] {}
                    edge from parent
                    node[above] {$T^c$}
                    node[below]  {$0.91$}
                }
                child {
                    node[end, label=right:
                        {$P(B^c\cap T)=0.99\cdot0.09 = 0.0891$}] {}
                    edge from parent
                    node[above] {$T$}
                    node[below]  {$0.09$}
                }
            edge from parent         
                node[above] {$B^c$}
                node[below]  {$0.99$}
        };
    \end{tikzpicture}
\end{tcolorbox}

\question Calculate to 6 decimal places the probability of a woman who tests positive for mammography screening having breast cancer.
\begin{tcolorbox}\vspace{-0.5cm}
\begin{align*}
    P\left(B\;\middle|\;T\right) = \frac{P(B \cap T)}{P(B\cap T) + P(B^c \cap T)} = \frac{0.009}{0.009 + 0.0891} = \frac{0.009}{0.0981} = \frac{10}{109} \approx \textbf{0.091743}\text{ (6 d.p.)}
\end{align*}
\end{tcolorbox}

\question What is the probability that a women who tests positive for screening twice have breast cancer?
\begin{tcolorbox}
Assuming independent tests, and letting $T_1$ denote testing positive the first time and $T_2$ denoting testing positive the second time:
\begin{align*}
    P(B \cap T_1 \cap T_2) &= 0.01 \cdot 0.9 \times 0.9 = 0.0081\\
    P(T_1) = P(T_2) &= P(T) = 0.009 + 0.0891 = 0.0981 \\
    P\left(B\;\middle|\;T_1\cap T_2\right) &= \frac{P(B \cap T_1 \cap T_2)}{P(T_1) P(T_2)} = \frac{0.0081}{0.0981^2} \approx \textbf{0.841680} \text{ (6 d.p.)}
\end{align*}
\end{tcolorbox}

\end{questions}
\vspace{-0.5cm}
\newpage
\section{Article 2}
\begin{questions}
\question Mickey currently had \$400 dollars in his prize pool. What would you advise him to do?
\begin{tcolorbox}
Representing $W$ as the amount of money in Mickey's pool after spinning the wheel and $\$m$ as his initial prize pool. The distribution of $W$ is:
\vspace{-0.2cm}
\begin{table}[H]
    \centering
    \begin{tabular}{|c|c|c|c|c|c|}
        \hline $w$ & $m + 50$ & $m+100$ & $m+500$ & $m/2$ & 0  \\
        \hline $P(W=w)$ & 0.5 & 0.25 & $1/12$ & $1/12$& $1/12$ \\
        \hline
    \end{tabular}
\end{table}
\vspace{-1cm}
\begin{align*}
    E(W) &= \frac{1}{2}\left(m+50\right)+\frac{1}{4}\left(m+100\right)+\frac{1}{12}\left(m+500\right)+\frac{1}{12}\frac{m}{2} \\
    &= \left(\frac{1}{2}+\frac{1}{4}+\frac{1}{12}+\frac{1}{24}\right)m + 25 + 25 + \frac{500}{12} = \frac{7}{8} m + \frac{1100}{12}
\end{align*}
When $n = 400$, $E(W) = 350 + \frac{1100}{12} > 400$, thus he should spin.
\end{tcolorbox}
\question Calculate $M$ such that one should only spin the wheel if they have less than \$$M$.
\begin{tcolorbox}
\begin{align*}
    m < \frac{7}{8} m + \frac{1100}{12} &\implies m < 8 \times \frac{1100}{12} \implies M = \textbf{733.333}\text{ (3 d.p.)}
\end{align*}
\end{tcolorbox}
\end{questions}
\section{Reflection and Analysis}

\end{document}